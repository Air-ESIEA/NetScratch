\documentclass[14pt,a4paper]{extarticle}
\author{Valentin MICHEL}
\date{15 Octobre 2019}
\usepackage{graphicx}
%\pagenumbering{gobble} 
%\makeatletter
%\renewcommand{\@seccntformat}[1]{}
%\makeatother

\title{Analyse de Besoins \\ Projet NetScratch}
\begin{document}
\maketitle{}
\center{\includegraphics[]{../../supertest.jpg}}

\break
\section{Étude de l'existant}
Le réseau d'IN'TECH est encore en chantier après la mise en service de l'infrastructure au cause des projets joints du semestre dernier.
Après une demande du service informatique du Groupe ESIEA, les anciens réseaux de TP d'IN'TECH, les réseaux Semestres ont été coupés. IN'TECH est aujourd'hui sans réseau de TP viable.
Dans ce contexte, les efforts actuels de la gouvernance informatique du parc IN'TECH cherche a déployer des réseaux TP utilisables pour les étudiants et leurs permettre de travailler.
C'est dans ce contexte que l'équipe NetScratch cherche à déployer des outils améliorant la QoS (qualité de service).
\section{Besoin}
Avec cette nouvelle infrastructure, un besoin d'optimisation de la bande passante apparait pour conserver une qualité de service satisfaisante. 
Nous avons identifiés deux points d'optimisation de bande passante :
\begin{itemize}
	\item{Mises à jour Système}
	\item{Publicité sur internet}
\end{itemize}
\subsection{Mises à jour système}
Les TP peuvent nécessiter des logiciels connu à l'avance par le professeur. Pour le bon déroulement du TP, chaque étuidant est supoosé avoir ces logiciels installés et à jour.
De plus, les mises à jour des divers systèmes d'expoitations utilisés au sein de l'école peuevent consommer une quantité importante de bande passante.
Les logiciels étant tous identiques, il est sub-optimal de les télécharger pour chaque machine depuis un serveur distant : un server de cache permet de limiter la quantité de téléchargement effectués sur le réseau. En effet, chaque machine télécharge les paquets sur le réseau local.
\subsection{Publicité}
Les pages servie sur Internet comportent souvent de la publicité. 
Le téléchargement de ces publicités consomme de la bande passante. En bloquant ces publicités à l'échelle du réseau, l'expérience utilisateur s'en trouve améliorée.
.
\section{Besoin fonctionnels}
Pour mettre en place ces deux améliorations du réseau, un serveur DNS est requis. Des techniques d'usurpation DNS devront être déployer pour forcer les client du réseau à utiliser le serveur de cache, et également pour le blocage de publicité, en prétendant être les serveurs de mises à jour, ou ceux des annonceurs publicitaires.
\end{document}
