\documentclass[14pt,a4paper]{extarticle}
\author{Valentin MICHEL}
\date{15 Octobre 2019}
\usepackage{graphicx}
\newcommand{\windaube}{Microsoft Windows®}
\usepackage{hyperref}
\usepackage[document]{ragged2e}
\usepackage[a4paper, tmargin=3cm]{geometry}
\title{Cahier des Charges \\ Projet NetScratch \\ Mission QoS}

\begin{document}
\maketitle{}
\center{\includegraphics[]{../../supertest.jpg}}
\justify
\break
\tableofcontents
\break
\section{Contexte et Objectifs}
\subsection{Contexte du Projet}
Le réseau d'IN'TECH est encore en chantier après la mise en service de l'infrastructure suite aux projets joints du semestre dernier. \\\\
Après une demande du service informatique du Groupe ESIEA, les anciens réseaux de TP d'IN'TECH, les réseaux "Semestre", ont été coupés. IN'TECH est aujourd'hui sans réseau de TP viable.\\\\
La gouvernance informatique du parc IN'TECH cherche a déployer des réseaux TP utilisables pour les étudiants et leurs permettre de travailler.\\\\
C'est dans ce contexte que l'équipe NetScratch travaille à déployer des outils améliorant la QoS (qualité de service).
\subsection{Objectifs}
\begin{itemize}
    \item{Optimiser l'utilisation de la bande passante à des fins de mises à jour de systèmes d'exploitation}
    \item{Bloquer la publicité sur le réseau IN'TECH}
    \item{Forcer les clients linux à utiliser les serveur de caches}
    \item{Automatiser le déploiement de serveur de fichier pour les logiciels de TP, TD...}
\end{itemize}
\break
\section{Contraintes}
\begin{itemize}
    \item{Ce projet devra être opérationnel pour la fin du semestre 4 IN'TECH.}
    \item{Ce projet ne doit pas interférer avec l'expérience des utilisateurs IN'TECH. En cas d'avarie des services déployés, les utilisateurs ne doivent pas être impactés.}
\end{itemize}
\section{Livrables}
\subsection{Livrable techniques}
\begin{itemize}
    \item{Serveur DNS Pi-Hole configuré}
    \item{Serveur de cache pour Ubuntu}
    \item{Serveur de cache pour Debian}
    \item{Serveur de cache pour CentOS / RHEL}
    \item{Serveur de MAJ \windaube WSUS}
    \item{Scripts de déployement automatique de serveur de fichier}
    \item{IHM pour les scripts de déployement}
\end{itemize}
\subsection{Livrable documentaires}
\begin{itemize}
    \item{Fiche PI}
    \item{Charte de Projet}
    \item{Cahier des Charges}
    \item{Documents d'exploitation}
    \item{Spécifications}
    \item{Cahier de Recette}
\end{itemize}
\break
\section{Moyens Disponibles}
\subsection{Matériel}
\begin{itemize}
    \item{Ordinateurs personnels des membres du projet}
    \item{Himalaya, l'infrastructure mutualisée d'IN'TECH}
\end{itemize}
\subsection{Logiciel}
\begin{itemize}
    \item{Python}
    \item{Ansible}
    \item{nginx}
    \item{apt-cacher}
    \item{VMware Workstation, VSphère}
    \item{ftp}
    \item{debian}
    \item{Pi-Hole}
    \item{squid}
\end{itemize}

\end{document}
