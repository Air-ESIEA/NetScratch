\documentclass[14pt,a4paper]{extarticle}
\author{Valentin MICHEL}
\date{15 Octobre 2019}
\usepackage{graphicx}
\usepackage{hyperref}
\usepackage[document]{ragged2e}
\usepackage[a4paper, tmargin=3cm]{geometry}
%\pagenumbering{gobble} 
%\makeatletter
%\renewcommand{\@seccntformat}[1]{}
%\makeatother

\title{Charte de Projet \\ Projet NetScratch}
\begin{document}
\maketitle{}
\center{\includegraphics[]{../../supertest.jpg}}
\justify
\break
\tableofcontents
\break
\section{Cadrage du projet}
\subsection{Délais de Réalisations}
Le projet s'étend sur la quasi totalité d'un semestre IN'TECH, soit de la première semaine d'octobre 2019 à fin janvier 2020.

\subsection{Directeur de Projet}
\subsubsection{QoS -IN'TECH}
Le directeur de projet pour la mission QoS est David Huet, professeur référent de la filière Système et Réseau.
\subsubsection{Asso}
Le directeur de projet pour la mission Asso est Lucas Grelaud, étudiant en 5ème année CFA à l'ESIEA

\break
\subsection{Chef de Projet et Équipe de réalisation}
L'équipe NetScratch est constituée de 3 membres :
\begin{itemize}
    \item{\textbf{Valentin Michel} - \textit{Chef de Projet}}
    \item{Ismaël Nimzil}
    \item{Alphène Ekofo Bosako}
\end{itemize}

\subsection{Client}
\subsubsection{QoS -IN'TECH}
Le client du projet NetScratch - QoS est la DSI d'IN'TECH
\subsubsection{Asso}
Les clients du projet NetScratch - Asso sont les associations techniques du Groupe ESIEA

\break
\section{Gestion du Reporting}
Le reporting du projet sera effectué toutes les semaines, lors du dernier créneau PI.
Il sera global pour les deux partie et sera envoyé aux deux directeurs de projet.
Les éventuelles informations confidentielles seront envoyées séparément
\section{Calendrier}
Le projet est découpé en 4 itérations, dont les dates sont encore à fixer :
\begin{itemize}
    \item{Itération 1 : Avant Projet - 7 novembre 2019}
    \item{Itération 2 : Installation et configuration}
    \item{Itération 3 : Automatisation}
    \item{Itération 4 : Recette}
\end{itemize}
Le cadrage de projet et la tenue de calendrier se fera avec l'outil \href{https://trello.com/b/XQZtrqGi}{Trello} \\
Ce dernier sera mis à jour au minimum une fois par semaine, le vendredi.
\section{Gestion de la Documentation}
La documentation est hébergé sur le google drive du projet : \href{https://drive.google.com/drive/folders/1G7gVQl3_Hf86D3YLuazpA_pVGj7RUXwx}{Drive}
Les fichiers plats seront sur Github : \href{https://github.com/daed4lus/NetScratch/}{GitHub}, ainsi que tout les fichiers de configuration

\break
\section{Sauvegardes}
\subsection{QoS -IN'TECH}
Des sauvegardes seront effectuées toutes les semaines
\subsection{Asso}
Par demande client, les sauvegardes seront effectuées une fois par semaine, sur les supports possibles. Les fichiers seront sauvegarder sur Github (voir Gestion de la Documentation)
\end{document}
